\section{Conclusión}
Habiendo culminado el trabajo podemos efectuar las siguientes conclusiones:
%=============================================
 \subsection{Un problema de ausentism}    
\begin{itemize}
    \item El problema se simplifica al ordenar a los empleados por el horario de comienzo de su turno de forma ascendente.
    \item Si algún empleado se cruzó con algún otro tuvo que haber sido después de que llegó y antes se vaya.
    \item Como la complejidad de nuestro algoritmo es acotada por la complejidad de ordenar los elementos, si estos cumpliesen requisitos para poder realizar un ordenamiento no comparativo, podría mejorar el tiempo de ejecución de nuestro programa. 
    
\end{itemize}
%=============================================
 \subsection{Una nueva regulación industrial}    
\begin{itemize}
    \item Es posible identificar el elemento mayoritario de un conjunto consultando cada elemento del mismo unas pocas veces.
    \item No es necesario ordenar el conjunto para solucionar el problema.
    \item El algoritmo encontrado escala linealmente con la cantidad de elementos del conjunto --i.e., es $\mathcal{O}(n)$--.
    \item El algoritmo es simple y poderoso: si existe un elemento mayoritario, lo va a hallar. 
    \item Si no hay un elemento mayoritario en el conjunto, el algoritmo puede obtener un candidato --a elemento mayoritario-- incorrecto. 
    \item Por esto, si con el algoritmo se obtiene un candidato, hay que recorrer el conjunto una última vez para corroborar que ese elemento sea efectivamente el que ocupa más de la mitad del conjunto. Esto no afecta el orden temporal del algoritmo.
\end{itemize}
%=============================================
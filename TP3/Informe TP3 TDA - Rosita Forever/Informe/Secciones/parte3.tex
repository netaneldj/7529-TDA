\section{Un poco de teoría}
%=============================================
\subsection{Definiciones}
Defina y explique (si es necesario con ejemplos) qué significa que un problema sea P, NP, NP-Completo y NP-Hard
%=============================================

\subsection{Problema}
Tenemos un problema A, un problema B y una caja negra NA y NB que resuelven el problema A y B respectivamente. Sabiendo que B es NP
%=============================================
\subsubsection{I}
Qué podemos decir de A si utilizamos NA para resolver el problema B (asumimos que la reducción realizada para adaptar el problema B al problema A es polinomial)
%=============================================
\subsubsection{II}
Qué podemos decir de A si utilizamos NB para resolver el problema A (asumimos que la reducción realizada para adaptar el problema A al problema B es polinomial)\newline

Decimos que A es igual o menos complejo que B.
%=============================================

\subsubsection{III}
Qué pasa con los puntos anteriores si no conocemos la complejidad de B, pero sabemos que A es P?

\begin{enumerate}
    \item Decimos que A es igual o más complejo que B y ambos se pueden resolver de forma polinómica.
    \item Decimos que B es igual o más complejo que A y no necesariamente B se resuelve de forma polinómica.
\end{enumerate}
%=============================================


\section{Anexo División de Bienes}

Una de las parejas más ricas del mundo está pasando por un proceso de divorcio. Entre sus bienes cuentan con propiedades, autos, motos, estampillas raras y otros coleccionables. Como no se ponen de acuerdo en la manera de dividirlos, el juez ha dictaminado que un tasador ponga valor a cada bien y luego se haga una partición por valores iguales. El juez nos pide que elaboremos un algoritmo que en forma eficiente haga este trabajo.\newline
%============================================

El pedido se puede plantear como un problema \textit{NUMBER-PARTITION}; dado un conjunto $C=\{w_{1}, w_{2}, ..., w_{n}\}$, donde cada $w_{i}$ está asociado al precio de cada bien, se desea determinar si existe un subconjunto W de C tal que la suma de sus elementos equivalga a la suma de los elementos restantes en C.

\begin{equation}
    \sum_{w\in W} w  = \sum_{w\in\overline{W}} w
\end{equation}

Para que un problema X sea NP-Completo, debe cumplirse que: \\
(1) $X$ sea NP\\
(2) cualquier problema NP-Completo $Y$ puede reducirse a $X$, en tiempo polinomial\\
(3) $X$ tenga una solución si y sólo si $Y$ tiene solución

\subsection{Demostración NP}

Corroborar que dos particiones de un conjunto sean solución de este problema es muy simple.  Sólo hay que verificar que la suma total de sus elementos sea equivalente. 
Por lo tanto, verificarlo es $\mathcal{O}(n)$, siendo n la cantidad de elementos del subconjunto más poblado. 
Entonces, al poder verificar la solución en tiempo polinomial, se puede asegurar que este problema es NP.

\subsection{Reducción del problema SUBSET-SUM}
Un conocido problema NP-Completo es el problema de la suma de los subconjuntos. 
Éste se puede describir como: Dado un conjunto de enteros $A$, ¿existe un subconjunto $S \subset A$, tal que la suma de sus elementos sea exactamente $k$?

En términos matemáticos:

\begin{gather*}
    A = \{w_{1}, w_{2}, ..., w_{n}\}, \sum_{w \in A} w_{i} = s \\
    \exists S \subset A  / \sum_{w \in S} w_{i} = k ?
\end{gather*}
Lo que se desea corroborar ahora es si \textit{SUBSET-SUM} $ \leq_{p} $ \textit{NUMBER-PARTITION}.
A tales efectos, se procederá a adaptar el problema \textit{SUBSET-SUM} para que sea resoluble a través de \textit{NUMBER-PARTITION}. 

Sea $s$ la suma de los miembros de $A$ y $A' = A \cup \{s-2k\}$ un conjunto definido, donde k es el valor objetivo previamente mencionado. Se aceptará que $A$ tiene solución si y sólo si \textit{NUMBER-PARTITION} acepta $A'$.

\begin{equation}
    A' = \{w_{1}, w_{2}, ..., w_{n}, s-2k\}, \sum_{w \in A'} = s + (s-2k) = 2s-2k
\end{equation}

Se desea probar que
$\{A,k\}$ $\in $ \textit{SUBSET-SUM} $\Leftrightarrow$ $\{A'\}$ $\in$ \textit{NUMBER-PARTITION}. Se hará esto en ambos sentidos.

\subsubsection{$\{A,k\}$ $\in $ \textit{SUBSET-SUM} $\Rightarrow$ $\{A'\}$ $\in$ \textit{NUMBER-PARTITION}}
 Si existe un conjunto de números en $A$ que sumen $k$, entonces el resto de los números sumarán $s-k$. 
\begin{gather*}
    S \subset A  / \sum_{w \in S} w = k \\
    \Rightarrow \sum_{w \in \overline{S}_{A}} w = \sum_{w \in A} w - \sum_{w \in S}w = s - k\\
\end{gather*}
Se llamará $\overline{S}$ al complemento de S en A. Entonces, es posible descomponer $A'$ de la siguiente forma (haciendo abuso de notación, pues $A$ y $A'$ son conjuntos casi idénticos).
    \begin{equation*}
    \begin{split}
        A' &= A \cup  \{s-2k\} \\
            &= \{\overline{S} \cup S\} \cup \{s-2k\} \\
            &= \overline{S}  \cup \{ S \cup \{s-2k\}\} \\
            &= B \cup \overline{B}_{A'}\\\              
    \end{split}
    \end{equation*}
Como $B \equiv \overline{S}$, la suma de sus elementos es "$s-k$". Luego, es posible calcular cuánto suma el conjunto de números restantes en A' (complemento de $B$ en $A'$).
    \begin{equation*}
    \begin{split}
    \sum_{w \in \overline{B}_{A'}} w & = \sum_{w \in A'} w - \sum_{w \in B} w \\
                            & =  2s-2k - (s-k) = s - k
    \end{split}
    \end{equation*}
    
Por lo tanto, existe una partición en dos de $A'$ ($B$ y $\overline{B}_{A'}$) tal que cada conjunto sume "$s-k$" (exactamente la mitad de la suma total de $A'$). 

\subsubsection{$\{A,k\}$ $\in $ \textit{SUBSET-SUM} $\Leftarrow$ $\{A'\}$ $\in$ \textit{NUMBER-PARTITION}} 
Si existe una partición en dos de $A'$ tal que cada una sume $s-k$ (la mitad de su suma), entonces una de ellas necesariamente contiene el número $s-2k$ que se agregó previamente. Al descartar del conjunto este valor, se obtiene un subconjunto de $A'$ cuya suma sea $k$.

\subsubsection{Corolario}
Un problema NP-Completo, como lo es el SUBSET-SUM, es reducible al problema del enunciado, NUMBER-PARTITION. Esta reducción es claramente en tiempo polinomial; para adaptar el conjunto "de entrada" sólo hay que incluir un valor (s-2k) para que pueda ser resuelto como una instancia de NUMBER-PARTITION. Además, para adaptar su salida sólo habría que identificar la partición con el valor que se agregó y descartarlo.\\
Teniendo todo esto en cuenta, se cumplen las 3 condiciones enunciadas y queda demostrado que NUMBER-PARTITION es NP-Completo porque, para poder resolverlo, se debe resolver el problema SUBSET-SUM, que es NP-completo.